\input texinfo  @c -*-texinfo-*-

@c (texinfo-insert-node-lines) will add @node to section/chapter
@c C-c C-u C-e will complete their names.
@c 
@c C-c C-u C-e  (texinfo-every-node-update) ---> complete node names and links
@c C-c C-u C-a  (texinfo-all-menus-update)  ---> update menu
@c C-c C-u m    (texinfo-master-menu)       ---> make master menu

@setfilename gxpman.info
@settitle GXP3 User's Guide
@iftex
@setchapternewpage on
@end iftex
@comment %**end of header


Copyright 2007 Kenjiro Taura (Read COPYRIGHT for detailed information.)


@titlepage
@title GXP3 User's Guide
@subtitle January 2007
@author Kenjiro Taura
@author
@author University of Tokyo
@author 7-3-1 Hongo Bunkyo-ku Tokyo, 113-0033 Japan
@page
@vskip 0pt plus 1filll
Copyright @copyright{} 2007 Kenjiro Taura
@end titlepage

@node Top, , (dir), (dir)

@ifhtml
@c HTML <link rel="STYLESHEET" href="gxpman.css" type="text/css" />
@c HTML <h1>GXP3 User Manual</h1>
@end ifhtml




@menu
@end menu





@chapter Getting Started


@section Prerequisites

In order to play with GXP, you need
@itemize
@item a Unix platform
@item Python interpreter
@end itemize
To do anything interesting with GXP, you should be able to remote
login (e.g., via SSH, RSH, QRSH, etc.) computers you want to use.

GXP has been developed on Linux and tested on Linux and FreeBSD.  It
has been tested with Python 1.5.2, 2.2.2, 2.3.4, and 2.4.2.  There are
high chances that it runs with later versions of Python.

Python is a very popular programming language that is installed on
most Linux platforms. For remote logins, it's very likely that it
accepts SSH logins. Thus there are chances that the above prerequisits
are already satisfied in your platform.

@section Installation

You can obtain the latest version (tarball) from:
@example
    http://www.logos.t.u-tokyo.ac.jp/gxp
@end example
and then unpack the tarball by something like:
@example
    tar zxvf gxp3.xx-yyyymmdd.tar.gz 
@end example
The exact file name (and the directory name) depends on the current
version of GXP. Perhaps you want to make a symlink @t{gxp3} to the
directory you obtained, or directly rename the directory to @t{gxp3}.

If you have cvs, it is recommended to use it for download.
@example
cvs -d :pserver:anonymous@@www.logos.ic.i.u-tokyo.ac.jp:/var/lib/cvs co gxp3
@end example

Either way, you will have directory called @t{gxp3}. Since GXP and
accompanying tools are all written in Python, you need no compilation.
Instead, all you need to do is either to:

@enumerate
@item make a symlink from anywhere in your path to @t{gxp3/gxpc}. For example,
@example
ln -s /absolute/path/to/gxp3/gxpc /usr/local/bin/gxpc
@end example
or,

@item add @t{gxp3} directory to your PATH environment variable. For example,
add the following line in your @t{.bashrc}.
@example
export PATH=$PATH:/absolute/path/to/gxp3
@end example
@end enumerate

In the former, it is important to make a symlink to, not a copy of,
@t{gxpc}. Otherwise it fails to find accompanying files under @t{gxp3}
directory.

To test your installation, type @t{gxpc} to your shell prompt and see
something like this.
@example
$ gxpc
gxpc: no daemon found, create one
/tmp/gxp-you/gxpsession-hongo019-you-2007-07-03-13-56-21-3561-19041263
@end example

@xref{Troubleshooting}, if you fail.


@chapter Tutorial

@section Testing Your Installation

The most basic function of GXP is to run commands on many hosts. To
this end, you first need to learn how to run commands and how to
acquire those hosts.

GXP is run from a regular shell prompt. For those who have used prior
versions of GXP, this has been changed in this version.

@section Running commands via e command
Given you put gxpc in your PATH in one way or another, type
the following to your shell prompt.
@example
$ gxpc e whoami
@end example
In this manual, @t{$} means a shell prompt, and is not a part of the input.

You will see something like the following.
@example
$ gxpc e whoami
gxpc: no daemon found, create one
tau
$
@end example

The @t{gxpc e} command lets GXP run whatever follows it, which is, in
this example, @t{whoami}.  As you can see in the first line, a GXP
``daemon'' is brought up, which takes the request to run @t{whoami}.
GXP daemon stays running background.  Thus, if you issue another gxpc
@t{e} command again, you will see result immediately this time.
@example
$ gxpc e whoami
tau
$ gxpc e uname
Linux
$
@end example

The daemon stays running even if you exit the shell or even logout.
If you exit or logout and then run your shell or login again, the
daemon should be still running.

@t{gxpc quit} is the command to terminate the daemon.
@example
$ gxpc quit
$
@end example
Issueing gxpc @t{e} command again, the daemon will bring up again, of course.
@example
$ gxpc e uname
gxpc: no daemon found, create one
Linux
@end example

A primitive way to check if the daemon is running is @t{ps} command. 
For example,
@example
$ ps w 
  PID TTY      STAT   TIME COMMAND
 9273 pts/1    S      0:00 /bin/bash --noediting -i
 9446 pts/1    S      0:00 python /home/tau/proj/gxp3/gxpd.py --no_stdin --redirect_stdout --redirect_stderr
 9486 pts/1    R      0:00 ps w
@end example
@t{gxpd.py} is the daemon running for you.

Or you can run @t{gxpc prompt} command. 
If no daemon is running, it does not print anything.
If a daemon is running, it
prints a short string indicating its status. For example,
@example
$ gxpc prompt
[1/1/1]
@end example
We will detail the meaning of these three numbers ([1/1/1]) later.
For now, remember @t{gxpc prompt} command as a way to check if the
daemon is running.

In general, @t{e} command will run any shell command on @i{all hosts
selected for execution.} In this example, you only have one host,
which is the local host you issued @t{gxpc} command on, and that node
is selected. @xref{Getting more hosts with use/explore commands},
for how to have more hosts and how to select a set of hosts for
execution.

@section Getting more hosts with use/explore commands

With GXP, you will probably want to use many hosts in parallel.  You
need to have a host you can remote-login. In this example, we assume
you are originally on host @t{hongo000} and you have another host,
@t{hongo001}, which you can remote-login from @t{hongo000} via
SSH. Furthermore, we assume you can do so without typing password
every time you login @t{hongo001}. 
??? SSH setup for GXP, if you are not familiar with necessary
SSH setup to do so. For those who already know it, 
you need to have public/private
key pairs and either (1) do not set the encryption passphrase of the private
key, or (2) use @t{ssh-agent} and @t{ssh-add} (@t{eval
`ssh-agent` && ssh-add}) and input the phassphrase to @t{ssh-add}, 
so SSH does not ask you it again.
Either way, you should have an environment
where something like the following silently succeeds without
your password/passphrase being asked by SSH client.
@example
$ ssh hongo001 hostname
hongo001
$
@end example
You may use remote-execution commands other than SSH, 
including RSH, QRSH, and QSUB as well as the local shell
(sh) to spawn multiple daemons on a single host. 
You can customize exact command lines used or even add
your own custom rsh-like command in the repertoire.
??? Explore Settings for Various Environment, 
for details.

Once you have such an environment, try the following to acquire hongo001.
@example
$ gxpc use ssh hongo000 hongo001
$ gxpc explore hongo001
reached : hongo001
$
@end example

The first line reads ``@t{gxpc} [can] @t{use} @t{ssh} [from]
@t{hongo000} [to login] @t{hongo001}.'' The second line instructs GXP to
go ahead and really login @t{hongo001}. If @t{hongo000} and @t{hongo001}
share your home directory, this command will not take much longer than
the regular SSH. After the explore command has finished, issue an
@t{e} command again. This time, you will see the command being
executed on the two hosts.
@example
$ gxpc e hostname
hongo000
hongo001
$
@end example

@t{gxpc stat} command shows hosts (daemons) connected.
@example
$ gxpc e hostname
hongo000
hongo001
$
@end example


If you want to grab a third host, say @t{hongo002}, try the following.
@example
$ gxpc use ssh hongo000 hongo002
$ gxpc explore hongo002
reached : hongo002
$ gxpc e hostname
hongo000
hongo001
hongo002
@end example

You could continue this way, issueing @t{use} command and then @t{explore}
command to get a single host at a time, but this is obviously not so
comfortable if you have hundred hosts, say @t{hongo000} -- @t{hongo099}.
The first trick to learn is that, both the second and the third
arguments to @t{use} command are actually @i{regular expressions} of
hostnames. Therefore, the following single line:
@example
$ gxpc use ssh hongo000 hongo0
@end example
says that ``@t{gxpc} [can] @t{use} @t{ssh} [from any host matching]
@t{hongo000} [to login any host matching] @t{hongo0}.''  Note that any
host that begins with @t{hongo0} matches the regular expression
@t{hongo0}.

Second, explore command can take multiple hostnames in a single
command. For example, after the above command, you may grab three 
hosts with a single stroke by:
@example
$ gxpc explore hongo003 hongo004 hongo005
reached : hongo003
reached : hongo004
reached : hongo005
$
@end example

By default, GXP won't grab the same host multiple times, so issuing
the following once again will have no effect.
@example
$ gxpc explore hongo001
$
@end example

If you have many hosts, this is still painful. An even better way
is to use a special notation @t{[[@i{xxx}-@i{yyy}]]}, which represents
a set of numbers between @i{xxx} and @i{yyy} (inclusive). So, 
@example
$ gxpc explore hongo[[000-014]]
$
@end example
is equivalent to 
@example
$ gxpc explore hongo000 hongo001 hongo002 ... hongo014
$
@end example
If you wish to exclude some hosts from the range, use a notation
@t{;}@i{nnn} or @t{;}@i{nnn}@t{-}@i{mmm}. For example,
@example
$ gxpc explore hongo[[000-014;007]]
$
@end example
will explore hongo000 ... hongo014 except for hongo007.
@example
$ gxpc explore hongo[[000-014;003-006]]
$
@end example
will explore hongo000 ... hongo014 except for hongo003-hongo006.

Since GXP gracefully handles (ignores) dead or non-existent hosts, you
normally do not have to exclude every single non-working hosts this
way, but doing so is sometimes useful to make explore faster.

Instead of typing hostnames in a command line, you may have a file
that lists targets and give it to explore command by 
@t{--targetfile} or the shorter @t{-t} option.
Let's say you have a file like the following.
@example
$ cat my_targets
hongo[[000-014]]
kashiwa[[000-009]]
edo[[000-012]]
@end example
Then
@example
$ gxpc explore @t{-t my_targets}
@end example
is equivalent to
@example
$ gxpc explore hongo[[000-014]] kashiwa[[000-009]] edo[[000-012]]
@end example

Now you reached 15 hosts in total. The @t{stat} command shows 
hosts (daemons) connected.

@example
$ gxpc stat
/tmp/gxp-tau/gxpsession-hongo000-tau-2007-01-27-23-15-04-29004-97439271
hongo000 (= hongo000-tau-2007-01-27-23-15-04-29004)
 hongo007 (= hongo007-tau-2007-01-27-23-16-30-24842)
 hongo001 (= hongo001-tau-2007-01-27-23-16-27-1205)
 hongo013 (= hongo013-tau-2007-01-27-23-48-28-20179)
 hongo014 (= hongo014-tau-2007-01-27-23-48-14-11435)
 hongo005 (= hongo005-tau-2007-01-27-23-16-30-9412)
 hongo006 (= hongo006-tau-2007-01-27-23-16-28-24083)
 hongo003 (= hongo003-tau-2007-01-27-23-16-29-16040)
 hongo002 (= hongo002-tau-2007-01-27-23-16-28-22077)
 hongo008 (= hongo008-tau-2007-01-27-23-48-12-4070)
 hongo012 (= hongo012-tau-2007-01-27-23-48-14-32519)
 hongo010 (= hongo010-tau-2007-01-27-23-48-14-12066)
 hongo011 (= hongo011-tau-2007-01-27-23-48-15-15126)
 hongo009 (= hongo009-tau-2007-01-27-23-48-13-21582)
 hongo004 (= hongo004-tau-2007-01-27-23-16-31-8776)
$
@end example

The first line shows the name of @i{a session file}, which is
explained later (??).  Below
@t{hongo000}, @t{hongo001}--@t{hongo014} are indented by a single space,
which means @t{hongo000} issued logins to all these hosts. This
happened because we have previously said
@example
$ gxpc use ssh hongo000 hongo0
@end example
indicating that (only) @t{hongo000} can login @t{hongo0}.

You can alternatively say
@example
$ gxpc use ssh hongo0 hongo0
@end example
indicating gxpc can use @t{ssh} from @t{any} host matching @t{hongo0}
to any host matching @t{hongo0}. This can be abbreviated to:
@example
$ gxpc use ssh hongo0
@end example
In this case, @t{gxpc} will try to reach these hosts in a more
load-balanced fashion. To see this let's quit and 
try it again from the beginning.

@example
$ gxpc quit
$ gxpc use ssh hongo0
gxpc: no daemon found, create one
$ gxpc explore -h /etc/hosts hongo00 hongo01[0-4]
reached : hongo003
reached : hongo008
reached : hongo001
reached : hongo004
reached : hongo002
reached : hongo005
reached : hongo007
reached : hongo006
reached : hongo009
reached : hongo013
reached : hongo010
reached : hongo011
reached : hongo012
reached : hongo014
$ gxpc stat
/tmp/gxp-tau/gxpsession-hongo000-tau-2007-01-28-00-10-51-311-66768183
hongo000 (= hongo000-tau-2007-01-28-00-10-51-311)
 hongo006 (= hongo006-tau-2007-01-28-00-10-33-29696)
 hongo005 (= hongo005-tau-2007-01-28-00-10-35-13470)
 hongo003 (= hongo003-tau-2007-01-28-00-10-34-16875)
 hongo002 (= hongo002-tau-2007-01-28-00-10-33-27494)
 hongo004 (= hongo004-tau-2007-01-28-00-10-36-9479)
 hongo009 (= hongo009-tau-2007-01-28-00-10-33-21614)
 hongo008 (= hongo008-tau-2007-01-28-00-10-31-4102)
  hongo011 (= hongo011-tau-2007-01-28-00-10-34-15158)
  hongo012 (= hongo012-tau-2007-01-28-00-10-33-32551)
  hongo010 (= hongo010-tau-2007-01-28-00-10-33-12098)
  hongo013 (= hongo015-tau-2007-01-28-00-10-34-20211)
  hongo014 (= hongo014-tau-2007-01-28-00-10-34-11467)
 hongo007 (= hongo007-tau-2007-01-28-00-10-35-26002)
 hongo001 (= hongo001-tau-2007-01-28-00-10-32-1979)
@end example

The indentations indicate that @t{hongo008} issued logins to
@t{hongo010}--@t{hongo014}. @t{hongo000} issued logins to @t{hongo001}
--@t{hongo009}. In general, GXP daemons will form a tree.  By default,
a single node tries to keep the number of its children no more than
nine.

@section Introduction to SSH for GXP Users
todo

@section Quick Reference of Frequently Used Commands

@itemize
@item 
@example
@t{gxpc prompt}
@end example
will show a succinct summary of gxp status.  It is strongly
recommended to put `gxpc prompt 2> /dev/null` 
(note for the backquotes, not regular
quotes) as part of your shell prompt. For example, if you are a bash
user, put the following into your .bashrc.
@example
export PS1='\h:\W`gxpc prompt 2> /dev/null`% '
@end example
Then you will always see the succinct summary in your shell prompt. e.g.,
@example
$ export PS1='\h:\W`gxpc prompt 2> /dev/null`% '
$ gxpc
hongo:tmp[1/1/1]% 
@end example

'2> /dev/null' is to make gxpc silently exit if should be there any
error.

@item 
@example
@t{gxpc use} @i{rsh-name} @i{src} @i{target}
@end example
tells gxp it can login
from @i{src} to @i{target} via @i{rsh-name}. 

@item
@example
@t{gxpc explore} @i{target} ...
@end example
attempts to login specified @i{target}'s by methods specified by @t{use} 
commands.
There is a convenient notation to represent a set of targets. e.g.,
@example
$ gxpc explore hongo[[000-012]]
@end example
is equivalent to
@example
$ gxpc explore hongo000 hongo001 hongo002 ... hongo012
@end example

@item 
@example
@t{gxpc e} @i{whatever}
@end example
will run a shell command @i{whatever} on 
all selected hosts. e.g.,

@item 
@example
@t{gxpc cd} @i{[directory]}
@end example
will change the current directory of all selected hosts. 

@item 
@example
@t{gxpc export} @i{VAR=VAL}
@end example
will set the environment variable for seqsequent commands. 

@item 
@example
@t{gxpc smask}
@end example
will select the hosts on which the last command succeeded
for the execution targets of subsequent commands.

@item 
@example
@t{gxpc rmask}
@end example
will reset the selected hosts to all hosts.

@item 
@example
@t{gxpc savemask} @i{name}
@end example
is similar to smask, but it remembers the set of selected hosts
for future reference (by @t{-m} option of @t{gxpc e} commands).

@item 
@example
@t{gxpc e} -m @i{name} @i{whatever}
@end example
is similar to @t{gxpc e} @i{whatever}, but runs @i{whatever} on
hosts that have been set by @i{name} by @t{gxpc savemask} @i{name}.

@item 
@example
@t{gxpc -m} @i{name} @t{cd} @i{[directory]}
@t{gxpc -m} @i{name} @t{export} @i{VAR=VAL}
@t{gxpc -m} @i{name} @t{e} @i{whatever}
@end example
@t{-m} can actually be written immediately after @t{gxpc}.

@item 
@example
@t{gxpc e} -h @i{hostname} @i{whatever}
@end example
is similar to @t{gxpc e} @i{whatever}, but runs @i{whatever} on
hosts whose names match regular expression @i{hostname}. 

@item 
@example
@t{gxpc e} -H @i{hostname} @i{whatever}
@end example
is similar to @t{gxpc e} @i{whatever}, but runs @i{whatever} on
hosts whose names do not match regular expression @i{hostname}. 

@item 
@example
@t{gxpc -h} @i{name} @t{cd} @i{[directory]}
@t{gxpc -h} @i{name} @t{export} @i{VAR=VAL}
@t{gxpc -h} @i{name} @t{e} @i{whatever}
@t{gxpc -H} @i{name} @t{cd} @i{[directory]}
@t{gxpc -H} @i{name} @t{export} @i{VAR=VAL}
@t{gxpc -H} @i{name} @t{e} @i{whatever}
@end example
Both @t{-h} and @t{-H} can actually be written immediately after @t{gxpc}.

@item
@example
alias e='gxpc e'
alias smask='gxpc smask'
...
@end example
Finally, it is recommended to put aliases to some frequently used
commands into your shell startup files, so that you do not need to
type 'gxpc' everytime.


@end itemize




@chapter Using GXP for Parallel Processing
todo

@chapter Command Reference

#include cmdref.tex

@chapter Tools Reference
todo

@chapter Troubleshooting
todo

@chapter Environment Variables Reference
todo

@chapter Key Stroke Reference
todo

@unnumberedsec Indices

@menu
* Function Index::              
* Variable Index::              
* Data Type Index::             
* Program Index::               
* Concept Index::               
@end menu

@unnumberedsec Function Index

@printindex fn

@unnumberedsec Variable Index

@printindex vr

@unnumberedsec Data Type Index

@printindex tp

@unnumberedsec Program Index

@printindex pg

@unnumberedsec Concept Index

@printindex cp


@contents

@bye
                                   

