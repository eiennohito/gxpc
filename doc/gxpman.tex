\input texinfo  @c -*-texinfo-*-

@c (texinfo-insert-node-lines) will add @node to section/chapter
@c C-c C-u C-e will complete their names.
@c 
@c C-c C-u C-e  (texinfo-every-node-update) ---> complete node names and links
@c C-c C-u C-a  (texinfo-all-menus-update)  ---> update menu
@c C-c C-u m    (texinfo-master-menu)       ---> make master menu

@setfilename gxpman.info
@settitle GXP3 User's Guide
@iftex
@setchapternewpage on
@end iftex
@comment %**end of header


Copyright 2007 Kenjiro Taura (Read COPYRIGHT for detailed information.)


@titlepage
@title GXP3 User's Guide
@subtitle January 2007
@author Kenjiro Taura
@author
@author University of Tokyo
@author 7-3-1 Hongo Bunkyo-ku Tokyo, 113-0033 Japan
@page
@vskip 0pt plus 1filll
Copyright @copyright{} 2007 Kenjiro Taura
@end titlepage

@node Top, Getting Started, (dir), (dir)

@ifhtml
@c HTML <link rel="STYLESHEET" href="gxpman.css" type="text/css" />
@c HTML <h1>GXP3 User Manual</h1>
@end ifhtml

@menu
* Getting Started::             
* Tutorial::                    
* Using GXP for Parallel Processing::  
* Command Reference::           
* Tools Reference::             
* Troubleshooting::             
* Environment Variables Reference::  
* Key Stroke Reference::        

@detailmenu
 --- The Detailed Node Listing ---

Getting Started

* Prerequisites::               
* Installation::                

Tutorial

* Testing Your Installation::   
* Running commands via e command::  
* Getting more hosts with use/explore commands::  
* Introduction to SSH for GXP Users::  
* Quick Reference of Frequently Used Commands::  

Command Reference

* cd::                          
* e::                           
* ep::                          
* explore::                     
* export::                      
* help::                        
* log_level::                   
* make::                        
* makectl::                     
* makeman::                     
* mapred::                      
* ping::                        
* popmask::                     
* pp::                          
* prof_start::                  
* smask::                       
* quit::                        
* reclaim::                     
* restoremask::                 
* rmask::                       
* rsh::                         
* set_max_buf_len::             
* show_explore::                
* showmasks::                   
* stat::                        
* trim::                        
* use::                         
* version::                     

Key Stroke Reference

* Indices::                     
* Function Index::              
* Variable Index::              
* Data Type Index::             
* Program Index::               
* Concept Index::               

Indices

* Function Index::              
* Variable Index::              
* Data Type Index::             
* Program Index::               
* Concept Index::               

@end detailmenu
@end menu

@node Getting Started, Tutorial, Top, Top
@chapter Getting Started


@menu
* Prerequisites::               
* Installation::                
@end menu

@node Prerequisites, Installation, Getting Started, Getting Started
@section Prerequisites

In order to play with GXP, you need
@itemize
@item a Unix platform
@item Python interpreter
@end itemize
To do anything interesting with GXP, you should be able to remote
login (e.g., via SSH, RSH, QSUB, etc.) computers you want to use.

GXP has been developed on Linux and tested on Linux and FreeBSD.  It
has been tested with Python 1.5.2, 2.2.2, 2.3.4, and 2.4.2.  There are
high chances that it runs with later versions of Python.

Python is a very popular programming language that is installed on
most Linux platforms. For remote logins, it's very likely that it
accepts SSH logins. Thus there are chances that the above prerequisits
are already satisfied in your platform.

@node Installation,  , Prerequisites, Getting Started
@section Installation

You can obtain the latest version (tarball) from sourceforge:
@example
    http://sourceforge.net/projects/gxp/
@end example
and then unpack the tarball by something like:
@example
    tar zxvf gxp3.xx.tar.bz2
@end example
The exact file name (and the directory name) depends on the current
version of GXP. Perhaps you want to make a symlink @t{gxp3} to the
directory you obtained, or directly rename the directory to @t{gxp3}.

If you have cvs, you can use it to obtain the latest copy under development.
@example
cvs -d :pserver:anonymous@@gxp.cvs.sourceforge.net:/cvsroot/gxp co gxp3
@end example

Either way, you will have directory called @t{gxp3}. Since GXP and
accompanying tools are all written in Python, you need no compilation.
Instead, all you need to do is either to:

@enumerate
@item make a symlink from anywhere in your path to @t{gxp3/gxpc}. For example,
@example
ln -s /absolute/path/to/gxp3/gxpc /usr/local/bin/gxpc
@end example
or,

@item add @t{gxp3} directory to your PATH environment variable. For example,
add the following line in your @t{.bashrc}.
@example
export PATH=$PATH:/absolute/path/to/gxp3
@end example
@end enumerate

In the former, it is important to make a symlink to, not a copy of,
@t{gxpc}. Otherwise it fails to find accompanying files under @t{gxp3}
directory.

To test your installation, type @t{gxpc} to your shell prompt and see
something like this.
@example
$ gxpc
gxpc: no daemon found, create one
/tmp/gxp-you/gxpsession-hongo019-you-2007-07-03-13-56-21-3561-19041263
@end example

@xref{Troubleshooting}, if you fail.


@node Tutorial, Using GXP for Parallel Processing, Getting Started, Top
@chapter Tutorial

@menu
* Testing Your Installation::   
* Running commands via e command::  
* Getting more hosts with use/explore commands::  
* Introduction to SSH for GXP Users::  
* Quick Reference of Frequently Used Commands::  
@end menu

@node Testing Your Installation, Running commands via e command, Tutorial, Tutorial
@section Testing Your Installation

The most basic function of GXP is to run commands on many hosts. To
this end, you first need to learn how to run commands and how to
acquire those hosts.

GXP is run from a regular shell prompt. For those who have used prior
versions of GXP, this has been changed in this version.

@node Running commands via e command, Getting more hosts with use/explore commands, Testing Your Installation, Tutorial
@section Running commands via e command
Given you put gxpc in your PATH in one way or another, type
the following to your shell prompt.
@example
$ gxpc e whoami
@end example
In this manual, @t{$} means a shell prompt, and is not a part of the input.

You will see something like the following.
@example
$ gxpc e whoami
gxpc: no daemon found, create one
tau
$
@end example

The @t{gxpc e} command lets GXP run whatever follows it, which is, in
this example, @t{whoami}.  As you can see in the first line, a GXP
``daemon'' is brought up, which takes the request to run @t{whoami}.
GXP daemon stays running background.  Thus, if you issue another gxpc
@t{e} command again, you will see result immediately this time.
@example
$ gxpc e whoami
tau
$ gxpc e uname
Linux
$
@end example

The daemon stays running even if you exit the shell or even logout.
If you exit or logout and then run your shell or login again, the
daemon should be still running.

@t{gxpc quit} is the command to terminate the daemon.
@example
$ gxpc quit
$
@end example
Issueing gxpc @t{e} command again, the daemon will bring up again, of course.
@example
$ gxpc e uname
gxpc: no daemon found, create one
Linux
@end example

A primitive way to check if the daemon is running is @t{ps} command. 
For example,
@example
$ ps w 
  PID TTY      STAT   TIME COMMAND
 9273 pts/1    S      0:00 /bin/bash --noediting -i
 9446 pts/1    S      0:00 python /home/tau/proj/gxp3/gxpd.py --no_stdin --redirect_stdout --redirect_stderr
 9486 pts/1    R      0:00 ps w
@end example
@t{gxpd.py} is the daemon running for you.

Or you can run @t{gxpc prompt} command. 
If no daemon is running, it does not print anything.
If a daemon is running, it
prints a short string indicating its status. For example,
@example
$ gxpc prompt
[1/1/1]
@end example
We will detail the meaning of these three numbers ([1/1/1]) later.
For now, remember @t{gxpc prompt} command as a way to check if the
daemon is running.

In general, @t{e} command will run any shell command on @i{all hosts
selected for execution.} In this example, you only have one host,
which is the local host you issued @t{gxpc} command on, and that node
is selected. @xref{Getting more hosts with use/explore commands},
for how to have more hosts and how to select a set of hosts for
execution.

@node Getting more hosts with use/explore commands, Introduction to SSH for GXP Users, Running commands via e command, Tutorial
@section Getting more hosts with use/explore commands

With GXP, you will probably want to use many hosts in parallel.  You
need to have a host you can remote-login. In this example, we assume
you are originally on host @t{hongo000} and you have another host,
@t{hongo001}, which you can remote-login from @t{hongo000} via
SSH. Furthermore, we assume you can do so without typing password
every time you login @t{hongo001}. 
@xref{Introduction to SSH for GXP Users}, if you are not familiar with necessary
SSH setup to do so. For those who already know it, 
you need to have public/private
key pairs and either (1) use @t{ssh-agent} and @t{ssh-add} (@t{eval
`ssh-agent` && ssh-add}) and input the phassphrase to @t{ssh-add}, 
so SSH does not ask you it again (recommended), or
(2) set the encryption passphrase of the private key empty.
Though the method (2) uses secure public-key protocol for
authentication, when your account on the host storing the private key
is compromised, the intruder may be able to access other hosts
via the private key.

Either way, you should have an environment
where something like the following silently succeeds without
your password/passphrase being asked by SSH client.
@example
$ ssh hongo001 hostname
hongo001
$
@end example
You may use remote-execution commands other than SSH, 
including RSH, QRSH, and QSUB as well as the local shell
(sh) to spawn multiple daemons on a single host. 
You can customize exact command lines used or even add
your own custom rsh-like command in the repertoire.
??? Explore Settings for Various Environment, 
for details.

Once you have such an environment, try the following to acquire hongo001.
@example
$ gxpc use ssh hongo000 hongo001
$ gxpc explore hongo001
reached : hongo001
$
@end example

The first line reads ``@t{gxpc} [can] @t{use} @t{ssh} [from]
@t{hongo000} [to login] @t{hongo001}.'' The second line instructs GXP to
go ahead and really login @t{hongo001}. If @t{hongo000} and @t{hongo001}
share your home directory, this command will not take much longer than
the regular SSH. After the explore command has finished, issue an
@t{e} command again. This time, you will see the command being
executed on the two hosts.
@example
$ gxpc e hostname
hongo000
hongo001
$
@end example

@t{gxpc stat} command shows hosts (daemons) connected.
@example
$ gxpc e hostname
hongo000
hongo001
$
@end example


If you want to grab a third host, say @t{hongo002}, try the following.
@example
$ gxpc use ssh hongo000 hongo002
$ gxpc explore hongo002
reached : hongo002
$ gxpc e hostname
hongo000
hongo001
hongo002
@end example

You could continue this way, issueing @t{use} command and then @t{explore}
command to get a single host at a time, but this is obviously not so
comfortable if you have hundred hosts, say @t{hongo000} -- @t{hongo099}.
The first trick to learn is that, both the second and the third
arguments to @t{use} command are actually @i{regular expressions} of
hostnames. Therefore, the following single line:
@example
$ gxpc use ssh hongo000 hongo0
@end example
says that ``@t{gxpc} [can] @t{use} @t{ssh} [from any host matching]
@t{hongo000} [to login any host matching] @t{hongo0}.''  Note that any
host that begins with @t{hongo0} matches the regular expression
@t{hongo0}.

Second, explore command can take multiple hostnames in a single
command. For example, after the above command, you may grab three 
hosts with a single stroke by:
@example
$ gxpc explore hongo003 hongo004 hongo005
reached : hongo003
reached : hongo004
reached : hongo005
$
@end example

By default, GXP won't grab the same host multiple times, so issuing
the following once again will have no effect.
@example
$ gxpc explore hongo001
$
@end example

If you have many hosts, this is still painful. An even better way
is to use a special notation @t{[[@i{xxx}-@i{yyy}]]}, which represents
a set of numbers between @i{xxx} and @i{yyy} (inclusive). So, 
@example
$ gxpc explore hongo[[000-014]]
$
@end example
is equivalent to 
@example
$ gxpc explore hongo000 hongo001 hongo002 ... hongo014
$
@end example
If you wish to exclude some hosts from the range, use a notation
@t{;}@i{nnn} or @t{;}@i{nnn}@t{-}@i{mmm}. For example,
@example
$ gxpc explore hongo[[000-014;007]]
$
@end example
will explore hongo000 ... hongo014 except for hongo007.
@example
$ gxpc explore hongo[[000-014;003-006]]
$
@end example
will explore hongo000 ... hongo014 except for hongo003-hongo006.

Since GXP gracefully handles (ignores) dead or non-existent hosts, you
normally do not have to exclude every single non-working hosts this
way, but doing so is sometimes useful to make explore faster.

Instead of typing hostnames in a command line, you may have a file
that lists targets and give it to explore command by 
@t{--targetfile} or the shorter @t{-t} option.
Let's say you have a file like the following.
@example
$ cat my_targets
hongo[[000-014]]
kashiwa[[000-009]]
edo[[000-012]]
@end example
Then
@example
$ gxpc explore @t{-t my_targets}
@end example
is equivalent to
@example
$ gxpc explore hongo[[000-014]] kashiwa[[000-009]] edo[[000-012]]
@end example

Now you reached 15 hosts in total. The @t{stat} command shows 
hosts (daemons) connected.

@example
$ gxpc stat
/tmp/gxp-tau/gxpsession-hongo000-tau-2007-01-27-23-15-04-29004-97439271
hongo000 (= hongo000-tau-2007-01-27-23-15-04-29004)
 hongo007 (= hongo007-tau-2007-01-27-23-16-30-24842)
 hongo001 (= hongo001-tau-2007-01-27-23-16-27-1205)
 hongo013 (= hongo013-tau-2007-01-27-23-48-28-20179)
 hongo014 (= hongo014-tau-2007-01-27-23-48-14-11435)
 hongo005 (= hongo005-tau-2007-01-27-23-16-30-9412)
 hongo006 (= hongo006-tau-2007-01-27-23-16-28-24083)
 hongo003 (= hongo003-tau-2007-01-27-23-16-29-16040)
 hongo002 (= hongo002-tau-2007-01-27-23-16-28-22077)
 hongo008 (= hongo008-tau-2007-01-27-23-48-12-4070)
 hongo012 (= hongo012-tau-2007-01-27-23-48-14-32519)
 hongo010 (= hongo010-tau-2007-01-27-23-48-14-12066)
 hongo011 (= hongo011-tau-2007-01-27-23-48-15-15126)
 hongo009 (= hongo009-tau-2007-01-27-23-48-13-21582)
 hongo004 (= hongo004-tau-2007-01-27-23-16-31-8776)
$
@end example

The first line shows the name of @i{a session file}, which is
explained later (??).  Below
@t{hongo000}, @t{hongo001}--@t{hongo014} are indented by a single space,
which means @t{hongo000} issued logins to all these hosts. This
happened because we have previously said
@example
$ gxpc use ssh hongo000 hongo0
@end example
indicating that (only) @t{hongo000} can login @t{hongo0}.

You can alternatively say
@example
$ gxpc use ssh hongo0 hongo0
@end example
indicating gxpc can use @t{ssh} from @t{any} host matching @t{hongo0}
to any host matching @t{hongo0}. This can be abbreviated to:
@example
$ gxpc use ssh hongo0
@end example
In this case, @t{gxpc} will try to reach these hosts in a more
load-balanced fashion. To see this let's quit and 
try it again from the beginning.

@example
$ gxpc quit
$ gxpc use ssh hongo0
gxpc: no daemon found, create one
$ gxpc explore -h /etc/hosts hongo00 hongo01[0-4]
reached : hongo003
reached : hongo008
reached : hongo001
reached : hongo004
reached : hongo002
reached : hongo005
reached : hongo007
reached : hongo006
reached : hongo009
reached : hongo013
reached : hongo010
reached : hongo011
reached : hongo012
reached : hongo014
$ gxpc stat
/tmp/gxp-tau/gxpsession-hongo000-tau-2007-01-28-00-10-51-311-66768183
hongo000 (= hongo000-tau-2007-01-28-00-10-51-311)
 hongo006 (= hongo006-tau-2007-01-28-00-10-33-29696)
 hongo005 (= hongo005-tau-2007-01-28-00-10-35-13470)
 hongo003 (= hongo003-tau-2007-01-28-00-10-34-16875)
 hongo002 (= hongo002-tau-2007-01-28-00-10-33-27494)
 hongo004 (= hongo004-tau-2007-01-28-00-10-36-9479)
 hongo009 (= hongo009-tau-2007-01-28-00-10-33-21614)
 hongo008 (= hongo008-tau-2007-01-28-00-10-31-4102)
  hongo011 (= hongo011-tau-2007-01-28-00-10-34-15158)
  hongo012 (= hongo012-tau-2007-01-28-00-10-33-32551)
  hongo010 (= hongo010-tau-2007-01-28-00-10-33-12098)
  hongo013 (= hongo015-tau-2007-01-28-00-10-34-20211)
  hongo014 (= hongo014-tau-2007-01-28-00-10-34-11467)
 hongo007 (= hongo007-tau-2007-01-28-00-10-35-26002)
 hongo001 (= hongo001-tau-2007-01-28-00-10-32-1979)
@end example

The indentations indicate that @t{hongo008} issued logins to
@t{hongo010}--@t{hongo014}. @t{hongo000} issued logins to @t{hongo001}
--@t{hongo009}. In general, GXP daemons will form a tree.  By default,
a single node tries to keep the number of its children no more than
nine.

@node Introduction to SSH for GXP Users, Quick Reference of Frequently Used Commands, Getting more hosts with use/explore commands, Tutorial
@section Introduction to SSH for GXP Users
todo

@node Quick Reference of Frequently Used Commands,  , Introduction to SSH for GXP Users, Tutorial
@section Quick Reference of Frequently Used Commands

@itemize
@item 
@example
@t{gxpc prompt}
@end example
will show a succinct summary of gxp status.  It is strongly
recommended to put `gxpc prompt 2> /dev/null` 
(note for the backquotes, not regular
quotes) as part of your shell prompt. For example, if you are a bash
user, put the following into your .bashrc.
@example
export PS1='\h:\W`gxpc prompt 2> /dev/null`% '
@end example
Then you will always see the succinct summary in your shell prompt. e.g.,
@example
$ export PS1='\h:\W`gxpc prompt 2> /dev/null`% '
$ gxpc
hongo:tmp[1/1/1]% 
@end example

'2> /dev/null' is to make gxpc silently exit if should be there any
error.

@item 
@example
@t{gxpc use} @i{rsh-name} @i{src} @i{target}
@end example
tells gxp it can login
from @i{src} to @i{target} via @i{rsh-name}. 

@item
@example
@t{gxpc explore} @i{target} ...
@end example
attempts to login specified @i{target}'s by methods specified by @t{use} 
commands.
There is a convenient notation to represent a set of targets. e.g.,
@example
$ gxpc explore hongo[[000-012]]
@end example
is equivalent to
@example
$ gxpc explore hongo000 hongo001 hongo002 ... hongo012
@end example

@item 
@example
@t{gxpc e} @i{whatever}
@end example
will run a shell command @i{whatever} on 
all selected hosts. e.g.,

@item 
@example
@t{gxpc cd} @i{[directory]}
@end example
will change the current directory of all selected hosts. 

@item 
@example
@t{gxpc export} @i{VAR=VAL}
@end example
will set the environment variable for seqsequent commands. 

@item 
@example
@t{gxpc smask}
@end example
will select the hosts on which the last command succeeded
for the execution targets of subsequent commands.

@item 
@example
@t{gxpc rmask}
@end example
will reset the selected hosts to all hosts.

@item 
@example
@t{gxpc savemask} @i{name}
@end example
is similar to smask, but it remembers the set of selected hosts
for future reference (by @t{-m} option of @t{gxpc e} commands).

@item 
@example
@t{gxpc e} -m @i{name} @i{whatever}
@end example
is similar to @t{gxpc e} @i{whatever}, but runs @i{whatever} on
hosts that have been set by @i{name} by @t{gxpc savemask} @i{name}.

@item 
@example
@t{gxpc -m} @i{name} @t{cd} @i{[directory]}
@t{gxpc -m} @i{name} @t{export} @i{VAR=VAL}
@t{gxpc -m} @i{name} @t{e} @i{whatever}
@end example
@t{-m} can actually be written immediately after @t{gxpc}.

@item 
@example
@t{gxpc e} -h @i{hostname} @i{whatever}
@end example
is similar to @t{gxpc e} @i{whatever}, but runs @i{whatever} on
hosts whose names match regular expression @i{hostname}. 

@item 
@example
@t{gxpc e} -H @i{hostname} @i{whatever}
@end example
is similar to @t{gxpc e} @i{whatever}, but runs @i{whatever} on
hosts whose names do not match regular expression @i{hostname}. 

@item 
@example
@t{gxpc -h} @i{name} @t{cd} @i{[directory]}
@t{gxpc -h} @i{name} @t{export} @i{VAR=VAL}
@t{gxpc -h} @i{name} @t{e} @i{whatever}
@t{gxpc -H} @i{name} @t{cd} @i{[directory]}
@t{gxpc -H} @i{name} @t{export} @i{VAR=VAL}
@t{gxpc -H} @i{name} @t{e} @i{whatever}
@end example
Both @t{-h} and @t{-H} can actually be written immediately after @t{gxpc}.

@item
@example
alias e='gxpc e'
alias smask='gxpc smask'
...
@end example
Finally, it is recommended to put aliases to some frequently used
commands into your shell startup files, so that you do not need to
type 'gxpc' everytime.


@end itemize




@node Using GXP for Parallel Processing, Command Reference, Tutorial, Top
@chapter Using GXP for Parallel Processing
todo

@node Command Reference, Tools Reference, Using GXP for Parallel Processing, Top
@chapter Command Reference

@menu
* cd::                          
* e::                           
* ep::                          
* explore::                     
* export::                      
* help::                        
* log_level::                   
* make::                        
* makectl::                     
* makeman::                     
* mapred::                      
* ping::                        
* popmask::                     
* pp::                          
* prof_start::                  
* smask::                       
* quit::                        
* reclaim::                     
* restoremask::                 
* rmask::                       
* rsh::                         
* set_max_buf_len::             
* show_explore::                
* showmasks::                   
* stat::                        
* trim::                        
* use::                         
* version::                     
@end menu

@node cd, e, Command Reference, Command Reference
@section cd
@vskip 5mm
@b{Usage:}
@example

  @t{gxpc} @t{cd} [@var{OPTIONS} ...] @var{DIRECTORY}

@end example
@vskip 5mm

@vskip 5mm
@b{Description:}


  Set current directory of the selected nodes to @var{DIRECTORY}.
Subsequent commands will start at the specified directory.
Options are the same as those of `@t{e}' command.

@node e, ep, cd, Command Reference
@section e, mw
@vskip 5mm
@b{Usage:}
@example

  @t{gxpc} @t{e}  [@var{OPTION} ...] @var{CMD}
  @t{gxpc} @t{mw} [@var{OPTION} ...] @var{CMD}
  @t{gxpc} @t{ep} [@var{OPTION} ...] @var{FILE}

@end example
@vskip 5mm

@vskip 5mm
@b{Description:}


  Execute the command on the selected nodes.

@vskip 5mm
@b{Options: }
@table @code
(for @t{mw} only):
@item   @t{--master} 'command'
    equivalent to @t{e} @t{--updown} '3:4:command' ...
  if @t{--master} is not given, it is equivalent to @t{e} @t{--updown} 3:4 ...

@end table
@vskip 5mm

@vskip 5mm
@b{Options: }
@table @code
(for @t{e}, @t{mw}, and @t{ep}):
@item   @t{--withmask},@t{-m} @var{MASK}
    execute on a set of nodes saved by @t{savemask} or @t{pushmask}
@item   @t{--withhostmask},@t{-h} @var{HOSTMASK}
    execute on a set of nodes whose hostnames match regexp @var{HOSTMASK}
@item   @t{--withhostnegmask},-H @var{HOSTMASK}
    execute on a set of nodes whose hostnames do not match regexp @var{HOSTMASK}
@item   @t{--withgupidmask},-g @var{HOSTMASK}
    execute on a set of nodes whose gupid (shown by @t{gxpc} @t{stat}) 
    match regexp @var{HOSTMASK}
@item   @t{--withgupidnegmask},-G @var{HOSTMASK}
    execute on a set of nodes whose gupid (shown by @t{gxpc} @t{stat}) 
    do not match regexp @var{HOSTMASK}
@item   @t{--withtargetmask},@t{-t} @var{HOSTMASK}
    execute on a set of nodes whose target name (shown by @t{gxpc} @t{stat}) 
    match regexp @var{HOSTMASK}
@item   @t{--withtargetnegmask},-T @var{HOSTMASK}
    execute on a set of nodes whose target name (shown by @t{gxpc} @t{stat}) 
    do not match regexp @var{HOSTMASK}
@item   @t{--up} FD0[:@var{FD1}]
    collect output from @var{FD0} of @var{CMD}, and output them to @var{FD1} of @t{gxpc}.
    if :@var{FD1} is omitted, it is treated as if @var{FD1} == @var{FD0}
@item   @t{--down} FD0[:@var{FD1}]
    broadcast input to @var{FD0} of @t{gxpc} to @var{FD1} of @var{CMD}.
    if :@var{FD1} is omitted, it is treated as if @var{FD1} == @var{FD0}
@item   @t{--updown} @var{FD1}:FD2[:@var{MASTER}]
    if :@var{MASTER} is omitted, collect output from @var{FD1} of @var{CMD},
    and broadcast them to @var{FD2} of @var{CMD}.
    if :@var{MASTER} is given, run @var{MASTER} on the local host, collect
    output from @var{FD1} of @var{CMD}, feed them to stdin of the @var{MASTER}.
    broadcast stdout of the @var{MASTER} to @var{FD1} of @var{CMD}.
@item   @t{--pty}
    assign pseudo tty for stdin/stdout/stderr of @var{CMD}

@end table
@vskip 5mm

By default,

@vskip 5mm
@itemize

@item  stdin of @t{gxpc} are broadcast to stdin of @var{CMD}
@item  stdout of @var{CMD} are output to stdout of @t{gxpc}
@item  stderr of @var{CMD} are output to stderr of @t{gxpc}

@end itemize
@vskip 5mm

This is as if `@t{--down} 0 @t{--up} 1 @t{--up} 2' are specified.  In this
case, stdout/stderr are block-buffered by default.  You may need
to do setbuf in your program or flush stdout/err, to display
@var{CMD}'s output without delay.  @t{--pty} overwrites this and turn them
to line-buffered (by default).  both stdout/err of @var{CMD} now goto
stdout of @t{gxpc} (they are merged).  @var{CMD}'s stdout/err should appear
as soon as they are newlined.

@vskip 5mm
@b{See Also:}
@example

  @t{smask} @t{savemask} @t{pushmask} @t{rmask} @t{restoremask} @t{popmask}

@end example
@vskip 5mm

@node ep, explore, e, Command Reference
@section ep
@vskip 5mm
@b{Usage:}
@example

  @t{gxpc} @t{e}  [@var{OPTION} ...] @var{CMD}
  @t{gxpc} @t{mw} [@var{OPTION} ...] @var{CMD}
  @t{gxpc} @t{ep} [@var{OPTION} ...] @var{FILE}

@end example
@vskip 5mm

@vskip 5mm
@b{Description:}


  Execute the command on the selected nodes.

@vskip 5mm
@b{Options: }
@table @code
(for @t{mw} only):
@item   @t{--master} 'command'
    equivalent to @t{e} @t{--updown} '3:4:command' ...
  if @t{--master} is not given, it is equivalent to @t{e} @t{--updown} 3:4 ...

@end table
@vskip 5mm

@vskip 5mm
@b{Options: }
@table @code
(for @t{e}, @t{mw}, and @t{ep}):
@item   @t{--withmask},@t{-m} @var{MASK}
    execute on a set of nodes saved by @t{savemask} or @t{pushmask}
@item   @t{--withhostmask},@t{-h} @var{HOSTMASK}
    execute on a set of nodes whose hostnames match regexp @var{HOSTMASK}
@item   @t{--withhostnegmask},-H @var{HOSTMASK}
    execute on a set of nodes whose hostnames do not match regexp @var{HOSTMASK}
@item   @t{--withgupidmask},-g @var{HOSTMASK}
    execute on a set of nodes whose gupid (shown by @t{gxpc} @t{stat}) 
    match regexp @var{HOSTMASK}
@item   @t{--withgupidnegmask},-G @var{HOSTMASK}
    execute on a set of nodes whose gupid (shown by @t{gxpc} @t{stat}) 
    do not match regexp @var{HOSTMASK}
@item   @t{--withtargetmask},@t{-t} @var{HOSTMASK}
    execute on a set of nodes whose target name (shown by @t{gxpc} @t{stat}) 
    match regexp @var{HOSTMASK}
@item   @t{--withtargetnegmask},-T @var{HOSTMASK}
    execute on a set of nodes whose target name (shown by @t{gxpc} @t{stat}) 
    do not match regexp @var{HOSTMASK}
@item   @t{--up} FD0[:@var{FD1}]
    collect output from @var{FD0} of @var{CMD}, and output them to @var{FD1} of @t{gxpc}.
    if :@var{FD1} is omitted, it is treated as if @var{FD1} == @var{FD0}
@item   @t{--down} FD0[:@var{FD1}]
    broadcast input to @var{FD0} of @t{gxpc} to @var{FD1} of @var{CMD}.
    if :@var{FD1} is omitted, it is treated as if @var{FD1} == @var{FD0}
@item   @t{--updown} @var{FD1}:FD2[:@var{MASTER}]
    if :@var{MASTER} is omitted, collect output from @var{FD1} of @var{CMD},
    and broadcast them to @var{FD2} of @var{CMD}.
    if :@var{MASTER} is given, run @var{MASTER} on the local host, collect
    output from @var{FD1} of @var{CMD}, feed them to stdin of the @var{MASTER}.
    broadcast stdout of the @var{MASTER} to @var{FD1} of @var{CMD}.
@item   @t{--pty}
    assign pseudo tty for stdin/stdout/stderr of @var{CMD}

@end table
@vskip 5mm

By default,

@vskip 5mm
@itemize

@item  stdin of @t{gxpc} are broadcast to stdin of @var{CMD}
@item  stdout of @var{CMD} are output to stdout of @t{gxpc}
@item  stderr of @var{CMD} are output to stderr of @t{gxpc}

@end itemize
@vskip 5mm

This is as if `@t{--down} 0 @t{--up} 1 @t{--up} 2' are specified.  In this
case, stdout/stderr are block-buffered by default.  You may need
to do setbuf in your program or flush stdout/err, to display
@var{CMD}'s output without delay.  @t{--pty} overwrites this and turn them
to line-buffered (by default).  both stdout/err of @var{CMD} now goto
stdout of @t{gxpc} (they are merged).  @var{CMD}'s stdout/err should appear
as soon as they are newlined.

@vskip 5mm
@b{See Also:}
@example

  @t{smask} @t{savemask} @t{pushmask} @t{rmask} @t{restoremask} @t{popmask}

@end example
@vskip 5mm

@node explore, export, ep, Command Reference
@section explore
@vskip 5mm
@b{Usage:}
@example

  @t{gxpc} @t{explore} [@var{OPTIONS}] @var{TARGET} @var{TARGET} ...

@end example
@vskip 5mm

@vskip 5mm
@b{Description:}


  Login target hosts specified by @var{OPTIONS} and @var{TARGET}.

@vskip 5mm
@b{Options:}
@table @code

@item   @t{--dry}
    dryrun. only show target hosts
@item   @t{--hostfile},@t{-h} @var{HOSTS_FILE}
    give known hosts by file
@item   @t{--hostcmd} @var{HOSTS_CMD}
    give known hosts by command output
@item   @t{--targetfile},@t{-t} @var{TARGETS_FILE}
    give target hosts by file
@item   @t{--targetcmd} @var{TARGETS_CMD}
    give target hosts by command output
@item   @t{--timeout} @var{SECONDS}
    specify the time to wait for a remote host's response
    until gxp considers it dead
@item   @t{--children_soft_limit} @var{N} (>= 2)
    control the shape of the @t{explore} tree. if this value is @var{N}, @t{gxpc}
    tries to keep the number of children of a single host no more than @var{N},
    unless it is absolutely necessary to reach requested nodes.
@item   @t{--children_hard_limit} @var{N}
    control the shape of the @t{explore} tree. if this value is @var{N}, @t{gxpc}
    keeps the number of children of a single host no more than @var{N}, in any event.
@item   @t{--target_prefix} @var{PATH}
    specify the directory in remote hosts in which gxp files are installed.
    default is ~/.gxp_tmp, meaning a temporary directory like
    ~/.gxp_tmp/RANDOM_NAME/gxp3 will be created and all files are installed
    there. automatically created if it does not exist. @t{use} @t{e}.g., /tmp/YOUR_NAME
    or something if you have ridicuously slow home directory.
@item   @t{--verbosity} @var{N} (0 <= @var{N} <= 2)
    set verbosity level (the larger the more verbose)
@item   @t{--set_default}
    if you set this option, options specified in this @t{explore} becomes the default.
    for example, if you say @t{--timeout} 20.0 and @t{--set_default}, timeout is set to
    20.0 in subsequent explores, even if you do not specify @t{--timeout}.
@item   @t{--reset_default}
    reset the default values set by @t{--set_default}.
@item   @t{--show_settings}
    show effective @t{explore} options, considering those given by command line and
    those specified as default values.

@end table
@vskip 5mm

Execution of an @t{explore} command will conceptually consist of the
following three steps.

(1) Known Hosts: Know names of existing hosts, either by
@t{--hostfile}, @t{--hostcmd}, or a default rule. These are called
'known hosts.' @t{-h} is an acronym of @t{--hostfile}.

(2) Targets: Extract login targets from known hosts. They are
extracted by regular expressions given either by @t{--targetfile},
@t{--targetcmd}, or directly by command line arguments. @t{-t} is
an acronym of @t{--targetfile}.

(3) @t{gxpc} will attempt to login these targets according to the
rules specified by `@t{use}' commands.

Known hosts are specified by a file using @t{--hostfile} option, or
by output of a command using @t{--hostcmd}. Formats of the two are
common and very simple. In the simplest format, a single file
contains a single hostname. For example,

@vskip 5mm
@example

   hongo001
   hongo002
   hongo004
   hongo005
   hongo006
   hongo007
   hongo008

@end example
@vskip 5mm

is a valid @var{HOSTS_FILE}. If you specify a command that outputs
a list of files in the above format, the effect is the same
as giving a file having the list by @t{--hostfile}. For example,

@vskip 5mm
@example

  @t{--hostcmd} 'for i in `seq 1 8` ; do printf "%03d\n" $i ; done'

@end example
@vskip 5mm

has the same effect as giving the above file to @t{--hostfile}.

The format of a @var{HOSTS_FILE} is actually a so-called @t{/etc/hosts}
format, each line of which may contain several aliases of the
same host, as well as their @var{IP} address. @t{gxpc} simply regards them
as aliases of a single host, wihtout giving any significance to
which columns they are in. Anything after `#' on each line is a
comment and ignored. Lines not containning any name, such as
empty lines, are also ignored.  The above simple format is
obviously a special case of this.

It is sometimes convenient to specify @t{/etc/hosts} as an argument
to @t{--hostfile} or to specify `ypcat hosts' as an argument to
@t{--hostcmd}. As a matter of fact, if you do not specify any of
@t{--hostfile}, @t{--hostcmd}, @t{--targetfile}, and @t{--targetcmd}, it is
treated as if @t{--hostfile} @t{/etc/hosts} is given.

Login targets are specified by a file using @t{--targetfile} option,
@t{--targetcmd} option, or by directly listing targets in the command
line. Format of them are common and only slightly different from
@var{HOSTS_FILE}.  The format of the list of targets in the command
line is as follows.

@vskip 5mm
@example

   @var{TARGET_REGEXP} [@var{N}] @var{TARGET_REGEXP} [@var{N}] @var{TARGET_REGEXP} [@var{N}] ...

@end example
@vskip 5mm

where @var{N} is an integer and @var{TARGET_REGEXP} is any string that cannot
be parsed as an integer. That is, it is a list of regular
expressions, each item of which may optionally be followed by an
integer. The integer indicates how many logins should occur to
the target matching @var{TARGET_REGEXP}. The following is a valid
command line.

@vskip 5mm
@example

  @t{gxpc} @t{explore} @t{-h} hosts_file hongo00

@end example
@vskip 5mm

which says you want to target all hosts beginning with hongo00,
among all hosts listed in hosts_file.  If, for example, you have
specified by `@t{use}' command that the local host can login these
hosts by ssh, you will reach hosts whose names begin with
hongo00.  If you instead say

@vskip 5mm
@example

  @t{gxpc} @t{explore} @t{-h} hosts_file hongo00 2

@end example
@vskip 5mm

you will get two processes on each of these hosts.

If you do not give any of @t{--targetfile}, @t{--targetcmd}, and command
line targets, it is treated as if a regular expression mathing
any string is given as the command line target. That is, all
known hosts are targets.

Format of targets_host is simply a list of lines each of which
is like the list of arguments just explained above. Thus, the
following is a valid @var{TARGETS_FILE}.

@vskip 5mm
@example

  hongo00 2
  chiba0
  istbs
  sheep

@end example
@vskip 5mm

which says you want to get two processes on each host beginning
with hongo00 and one process on each host beginning with chiba0,
istbs, or sheep. Just to illustrate the syntax, the same thing
can be alternatively written with different arrangement into
lines.

@vskip 5mm
@example

  hongo00 2 chiba0
  istbs sheep

@end example
@vskip 5mm

Similar to hosts_file, you may instead specify a command line
producing the output conforming to the format of @var{TARGETS_FILE}.

We have so far explained that target_regexp is matched against a
pool of known hosts to generate the actual list of targets.
There is an exception to this. If @var{TARGET_REGEXP} does not match
any host in the pool of known hosts, it is treated as if the
@var{TARGET_REGEXP} is itself a known host. Thus,

@vskip 5mm
@example

  @t{gxpc} @t{explore} hongo000 hongo001

@end example
@vskip 5mm

will login hongo000 and hongo001, because neither hosts_file nor
hosts_cmd hosts are given so these expressions obviously won't
match any known host. Using this rule, you may have a file that
explicitly lists all hosts and solely @t{use} it to specify targets
without using separate @var{HOSTS_FILE}. For example, if you have a
long @var{TARGETS_FILE} called targets like:

@vskip 5mm
@example

  abc000
  abc001
    ...
  abc099
  def000
  def001
    ...
  def049
  pqr000
  pqr001
    ...
  pqr149

@end example
@vskip 5mm

and say

@vskip 5mm
@example

  @t{gxpc} @t{explore} @t{-t} targets

@end example
@vskip 5mm

you say you want to get these 300 targets using whatever methods
you specified by `@t{use}' commands.

Unlike @var{HOSTS_FILE}, an empty line in @var{TARGETS_FILE} is treated as if
it is the end of file. By inserting an empty line, you can easily
let @t{gxpc} ignore the rest of the file. This rule is sometimes
convenient when targeting a small number of hosts within a
@var{TARGETS_FILE}.

Here are some examples.

1.

@vskip 5mm
@example

  @t{gxpc} @t{explore} @t{-h} hosts_file chiba hongo

@end example
@vskip 5mm

Hosts beginning with chiba or hongo in hosts_file 
become the targets.

2.

@vskip 5mm
@example

  @t{gxpc} @t{explore} @t{-h} hosts_file @t{-t} targets_file

@end example
@vskip 5mm

Hosts matching any regular expression in targets_file become
the targets.

3.

@vskip 5mm
@example

  @t{gxpc} @t{explore} @t{-h} hosts_file

@end example
@vskip 5mm

All hosts in hosts_file become the targets.  Equivalent to `@t{gxpc}
@t{explore} @t{-h} hosts_file .'  (`.' is a regular expression mathing
any non-empty string).

4.

@vskip 5mm
@example

  @t{gxpc} @t{explore} @t{-t} targets_file

@end example
@vskip 5mm

All hosts in targetfile become the targets. This is simiar to the
previous case, but the file format is different.  Note that in
this case, strings in targets_file won't be matched against
anything, so they should be literal target names.
     
5.

@vskip 5mm
@example

  @t{gxpc} @t{explore} chiba000 chiba001 chiba002 chiba003

@end example
@vskip 5mm

chiba000, chiba001, chiba002, and chiba003 become the targets.

6.

@vskip 5mm
@example

  @t{gxpc} @t{explore} chiba0

@end example
@vskip 5mm

Equivalent to `@t{gxpc} @t{explore} @t{-h} @t{/etc/hosts} chiba0' which is hosts
beginning with chiba0 in @t{/etc/hosts} become the targets. Useful
when you @t{use} a single cluster and all necessary hosts are listed
in that file.
     
7.

@vskip 5mm
@example

  @t{gxpc} @t{explore}

@end example
@vskip 5mm

Equivalent to `@t{gxpc} @t{explore} @t{-h} @t{/etc/hosts}' which is in turn
equivalent to `@t{gxpc} @t{explore} @t{-h} @t{/etc/hosts} .'  That is, all hosts
in @t{/etc/hosts} become the targets.  This will be rarely useful
because @t{/etc/hosts} typically includes hosts you don't want to
@t{use}.

@node export, help, explore, Command Reference
@section export
@vskip 5mm
@b{Usage:}
@example

  @t{gxpc} @t{export} @var{VAR}=@var{VAL}

@end example
@vskip 5mm

@vskip 5mm
@b{Description:}


  Set environment variable @var{VAR} to @var{VAL} on the selected nodes.
Options are the same as those of `@t{e}' command.

@node help, log_level, export, Command Reference
@section help
@vskip 5mm
@b{Usage:}
@example

  @t{gxpc} @t{help}
  @t{gxpc} @t{help} @var{COMMAND}

@end example
@vskip 5mm

@vskip 5mm
@b{Description:}


  Show summary of @t{gxpc} commands or a @t{help} on a specific @var{COMMAND}.

@node log_level, make, help, Command Reference
@section log_level, log_base_time
@vskip 5mm
@b{Usage:}
@example

  @t{gxpc} @t{log_level} @var{LEVEL}
  @t{gxpc} @t{log_base_time}

@end example
@vskip 5mm

@vskip 5mm
@b{Description:}


  Command @t{log_level} will set the log level of the selected nodes
to the specified @var{LEVEL}.  0 will write no logs. 2 will write many.
Command @t{log_base_time} will reset the time of the selected nodes
to zero.  Subsequent log entries will record the time relative to
this time.

@node make, makectl, log_level, Command Reference
@section make
@vskip 5mm
@b{Usage:}
@example

  @t{gxpc} @t{make} GNU-MAKE-ARGS [ -- GXPC-MAKE-ARGS ]

@end example
@vskip 5mm

@vskip 5mm
@b{Description:}


  Parallel and distributed @t{make}.

@vskip 5mm
@b{Options:}
@table @code


@end table
@vskip 5mm

@node makectl, makeman, make, Command Reference
@section makectl
@vskip 5mm
@b{Usage:}
@example

  @t{gxpc} @t{makectl} (leave|leave_now|join)

@end example
@vskip 5mm

@vskip 5mm
@b{Description:}


  @var{GXP} daemons who receive this command will leave, leave immediately,
or join the gxp @t{make} computation.
@vskip 5mm
@b{Options:}
@table @code


@end table
@vskip 5mm

@node makeman, mapred, makectl, Command Reference
@section makeman
@vskip 5mm
@b{Usage:}
@example

  @t{gxpc} @t{makeman}

@end example
@vskip 5mm

@vskip 5mm
@b{Description:}


  Generate command reference chapter of gxp manual.

@node mapred, ping, makeman, Command Reference
@section mapred
@vskip 5mm
@b{Usage:}
@example

  @t{gxpc} @t{mapred} GNU-MAKE-ARGS [ -- GXPC-MAKE-ARGS ]

@end example
@vskip 5mm

@vskip 5mm
@b{Description:}


  This is a map-reduce framework built on top of @var{GXP} @t{make}.
You can run map-reduce without writing any Makefile by yourself.
You specify various options in GNU-MAKE-ARGS in the form of 
var=val.  @var{A} simple example:

@vskip 5mm
@example

  @t{gxpc} @t{mapred} -j input=big.txt output=out.txt \
    mapper=./my_mapper reducer=./my_reducer

@end example
@vskip 5mm

You will find the '-n' option of @t{make} useful, since it tells you
which commands are going to be executed by this command line.

@vskip 5mm
@example

  @t{gxpc} @t{mapred} -n input=big.txt output=out.txt \
    mapper=./my_mapper reducer=./my_reducer

@end example
@vskip 5mm

@vskip 5mm
@b{Options:}
@table @code

You will probably want to specify at least the following.
  input=<input filename>     (default: "input")
  output=<output filename>   (default: "output")
  mapper=<mapper command>    (default: "ex_word_count_mapper")
  reducer=<reducer command>  (default: "ex_word_count_reducer")

@end table
@vskip 5mm

<input filename> and <output filename> are filenames.
<mapper command> is a command that reads anything from stdin and 
writes key-value pairs.
<reducer command> is a command that reads key-value pairs from stdin 
in sorted order and writes arbitrary final outputs.

You will frequently want to specify the following.
  n_maps=<number of map tasks>       (default: 4)
  n_reduces=<number of reduce tasks> (default: 2)
  reader=<reader command>            (default: "ex_line_reader")
  int_dir=<intermediate directory>   (default: "int_dir")
  keep_intermediates=y  will keep all intermediate files in int_dir
  small_step=y  will execute the entire computation in small steps
  dbg=y   equivalent to keep_intermediates=y small_step=y (useful for debugging)

More options are:
  partitioner=<partitioner command>  (default: "ex_partitioner")
  pre_reduce_sorter=<sort command>   (default: "sort")
  final_merger=<merge command>       (default: "cat")

pre_reduce_sorter is a command that runs before each reducer.
It takes in the command line mappers' output filenames and should
output to the stdout the sorted list of key-value pairs.
final_merger takes the filenames of all the reducers' output and
outputs the final result.


@node ping, popmask, mapred, Command Reference
@section ping
@vskip 5mm
@b{Usage:}
@example

  @t{gxpc} @t{ping} [@var{LEVEL}]

@end example
@vskip 5mm

@vskip 5mm
@b{Description:}


  Send a small message to the selected nodes and show some
information. The parameter @var{LEVEL} is 0 or 1. Default is 0.
This is useful to know the name and basic information about
all or some nodes. It is also useful to check the status (liveness)
of nodes and @t{trim} non-responding nodes.

@vskip 5mm
@b{See Also:}
@example

  @t{trim} @t{smask}

@end example
@vskip 5mm

@node popmask, pp, ping, Command Reference
@section popmask
@vskip 5mm
@b{Usage:}
@example

  @t{gxpc} @t{popmask}

@end example
@vskip 5mm

@vskip 5mm
@b{Description:}


  Pop the set of nodes on the top of the stack. The next entry
that is used to be referred to by name `1' will now be the top
of the stack, and thus become the default set of nodes selected
for execution.

@vskip 5mm
@b{See Also:}
@example

  @t{pushmask}

@end example
@vskip 5mm

@node pp, prof_start, popmask, Command Reference
@section pp
@vskip 5mm
@b{Usage:}
@example

  @t{gxpc} @t{pp} GNU-MAKE-ARGS [ -- GXPC-MAKE-ARGS ]

@end example
@vskip 5mm

@vskip 5mm
@b{Description:}


  This is a simple parameter parallel framework built on top of @var{GXP} @t{make}.
You can run a simple parameter-sweep type parallel applications without 
writing any Makefile by yourself.  You specify various options in GNU-MAKE-ARGS 
in the form of var=val.  @var{A} simple example:

@vskip 5mm
@example

  @t{gxpc} @t{pp} -j cmd='./my_cmd -a $(a) $(f) ' a="1 2 3 4" f="a.txt b.txt c.txt" @t{parameters}="a f"

@end example
@vskip 5mm

This will execute "./my_cmd -a $(a) $(f)" for all the 4x3=12 combinations
of a and f. @t{parameters}= is mandatory.  For all names listed in @t{parameters},
you need to specify at least one value.  You will find the '-n' option of 
@t{make} useful, since it tells you which commands are going to be executed by 
this command line.

@vskip 5mm
@example

  @t{gxpc} @t{pp} -n cmd='./my_cmd -a $(a) $(f) ' a="1 2 3" f="a.txt b.txt c.txt"

@end example
@vskip 5mm


@node prof_start, smask, pp, Command Reference
@section prof_start, prof_stop
@vskip 5mm
@b{Usage:}
@example

  @t{gxpc} @t{prof_start} @var{FILENAME}
  @t{gxpc} @t{prof_stop}

@end example
@vskip 5mm

@vskip 5mm
@b{Description:}


  Start/stop profiling of the selected nodes. Stats are saved to
the @var{FILENAME}.

@node smask, quit, prof_start, Command Reference
@section smask, savemask, pushmask
@vskip 5mm
@b{Usage:}
@example

  @t{gxpc} @t{smask}    [-]
  @t{gxpc} @t{savemask} [-] @var{NAME}
  @t{gxpc} @t{pushmask} [-]
  @t{e}.g.,
  @t{gxpc}  @t{e}  'uname | grep Linux'
  @t{gxpc}  @t{smask}

@end example
@vskip 5mm

@vskip 5mm
@b{Description:}


  All three commands have a common effect. That is to modify the
set of nodes that will execute subsequent commands.  When `-'
argument is not given, nodes that executed the last command and
succeeded are selected. With argument `-', nodes that executed
the last command and failed are set. The definition of sucess or
failure depends on commands, but in the case of `@t{e}' command, a
node is considered to succeed if the command exits with status
zero.

These commands can be used to efficiently choose the nodes
to execute subsequent commands by various criterion.

@vskip 5mm
@b{Examples:}
@example


@end example
@vskip 5mm

1.

@vskip 5mm
@example

  @t{gxpc}  @t{e}  'uname | grep Linux'
  @t{gxpc}  @t{smask}

@end example
@vskip 5mm

This will set the execution mask of Linux nodes.

2.

@vskip 5mm
@example

  @t{gxpc}  @t{e}  'which apt-get'
  @t{gxpc}  @t{smask}  -
  @t{gxpc}  @t{e}  hostname

@end example
@vskip 5mm

This will set the execution mask of nodes that do not have apt-get
command, and the last command will show their hostnames.

To see the effect of these commands, it is advised to include the
gxp3 directory in your @var{PATH}, and add something like the following
in your shell (bash) prompt, which will show the number of nodes
that succeeded the last command, the number of nodes that is
currently selected, and the number all nodes.

@vskip 5mm
@example

  @t{export} @var{PS1}='\h:\W`which gxp_prompt 1> /dev/null && gxp_prompt`% '

@end example
@vskip 5mm

With this, you can see the effect of these commands in shell
prompt.

@vskip 5mm
@example

  [66/66/66]% @t{e} 'uname | grep Linux'
  Linux
  Linux
  Linux
  Linux
  Linux
  Linux
  Linux
  Linux
  Linux
  Linux
  [10/66/66]% @t{gxpc} @t{smask}
  [10/10/66]% 

@end example
@vskip 5mm

In addition to setting the execution mask, @t{savemask} saves the set
of selected nodes with the specified name. Sets of nodes hereby
saved can be later used for execution by giving @t{--withmask} (@t{-m})
option. This is useful when your work needs several, typical set
of nodes to execute commands on. For example, you may save a
small number of nodes for test, all gateway nodes to compile
programs, all nodes within a particular cluster, and really all
nodes.

Command @t{pushmask} is similar to @t{savemask}, but the set is saved
onto a stack. The newly selected set of nodes are on the top of
the stack and named `0'.  Previously selected nodes are named by
the distance from the top.

@vskip 5mm
@b{See Also:}
@example

  @t{showmasks} @t{rmask} @t{restoremask} @t{popmask}

@end example
@vskip 5mm

@node quit, reclaim, smask, Command Reference
@section quit
@vskip 5mm
@b{Usage:}
@example

  @t{gxpc} @t{quit} [@t{--session_only}]

@end example
@vskip 5mm

@vskip 5mm
@b{Description:}


  Quit gxp session. By default, all daemons will exit.
If @t{--session_only} is given, daemons keep running and only the 
session will cease.

@node reclaim, restoremask, quit, Command Reference
@section reclaim
@vskip 5mm
@b{Usage:}
@example

  @t{gxpc} @t{reclaim} tid

@end example
@vskip 5mm

@vskip 5mm
@b{Description:}


  Command @t{reclaim} will @t{reclaim} task tid unconditionally.

@node restoremask, rmask, reclaim, Command Reference
@section restoremask
@vskip 5mm
@b{Usage:}
@example

  @t{gxpc} @t{restoremask} @var{NAME}

@end example
@vskip 5mm

@node rmask, rsh, restoremask, Command Reference
@section rmask
@vskip 5mm
@b{Usage:}
@example

  @t{gxpc} @t{rmask}

@end example
@vskip 5mm

@vskip 5mm
@b{Description:}


  Reset execution mask. Let all nodes execute subsequent commands.

@vskip 5mm
@b{See Also:}
@example

  @t{showmasks} @t{smask} @t{savemask} @t{restoremask} @t{pushmask} @t{popmask}

@end example
@vskip 5mm

@node rsh, set_max_buf_len, rmask, Command Reference
@section rsh
@vskip 5mm
@b{Usage:}
@example

  @t{gxpc} @t{rsh} [@var{OPTIONS}]
  @t{gxpc} @t{rsh} [@var{OPTIONS}] rsh_name
  @t{gxpc} @t{rsh} [@var{OPTIONS}] rsh_name rsh_like_command_template

@end example
@vskip 5mm

@vskip 5mm
@b{Description:}


  Show, add, or modify rsh-like command explore/use will recognize
  to the repertoire.
  'gxpc @t{rsh}' lists all configured rsh-like commands.
  'gxpc @t{rsh} rsh_name' shows the specified rsh-like command.
  'gxpc @t{rsh} rsh_name rsh_like_command_template' adds or modifies (if exist)
  the specified rsh-like command to @t{use} rsh_like_command_template.

@vskip 5mm
@example

  The rsh_name is used as the first parameter of 'use' command (@t{e}.g.,
  'gxpc @t{use} ssh src target' or 'gxpc @t{use} @t{rsh} <src> <target>').

@end example
@vskip 5mm

  By default, the following rsh-like commands are builtin.

@vskip 5mm
@example

  ssh, ssh_as, @t{rsh}, rsh_as, sh, sge, torque, and pbs.

@end example
@vskip 5mm

@vskip 5mm
@b{Options:}
@table @code

@item   @t{--full}
    when invoked as @t{gxpc} @t{rsh} @t{--full} (with no other args),
    show command lines of all available rsh-like commands.

@end table
@vskip 5mm

@vskip 5mm
@b{Examples:}
@example

1.
  @t{gxpc} @t{rsh} ssh
  (output) ssh : ssh -o 'StrictHostKeyChecking no' ... -A %target% %cmd%

@end example
@vskip 5mm

  This displays that an rsh-like command named 'ssh' is configured,
  and gxp understands that, to run a command a host via ssh, it should
  @t{use} a command:

@vskip 5mm
@example

         ssh -o 'StrictHostKeyChecking no' ... -A %target% %cmd%

@end example
@vskip 5mm

  with %target% replaced by a target name (normally a
  hostname) and %cmd% by whatever commands it wants
  to execute on the target.
  
2.   
  @t{gxpc} @t{rsh} ssh ssh -i elsewhere/id_dsa -o 'StrictHostKeyChecking no' -A %target% %cmd%

@vskip 5mm
@example

  This instructs gxp to @t{use} command line:

@end example
@vskip 5mm

     ssh -i elsewhere/id_dsa -o 'StrictHostKeyChecking no' -A %target% %cmd%

@vskip 5mm
@example

  when it uses ssh.  You can arbitrarily name a new rsh-like command. 
  For example, let's say on some hosts, ssh daemons listen on customized 
  ports (say 2000) and you need to connect to that port to login those 
  hosts, while connecting to the regular port to login others. Then you 
  first define a new rsh-like command.

@end example
@vskip 5mm
  
3.
  @t{gxpc} @t{rsh} ssh2000 ssh -p 2000 -o 'StrictHostKeyChecking no' -A %target% %cmd%

@vskip 5mm
@example

  And you specicy ssh2000 label to login those hosts, using 'use' command. @t{e}.g.,

@end example
@vskip 5mm

     @t{use} ssh2000 <src> <target>

@vskip 5mm
@b{See Also:}
@example

  @t{use} @t{explore}

@end example
@vskip 5mm


@node set_max_buf_len, show_explore, rsh, Command Reference
@section set_max_buf_len
@vskip 5mm
@b{Usage:}
@example

  @t{gxpc} @t{set_max_buf_len} @var{N}

@end example
@vskip 5mm

@vskip 5mm
@b{Description:}


  Set maximum internal buffer size of gxp daemons in bytes.
  default is 10KB and any value below the default is ignored. If no argument
  is given, @t{use} the default value.

@node show_explore, showmasks, set_max_buf_len, Command Reference
@section show_explore
@vskip 5mm
@b{Usage:}
@example

  @t{gxpc} @t{show_explore} @var{SRC} @var{TARGET}

@end example
@vskip 5mm

@vskip 5mm
@b{Description:}


  Show command used to @t{explore} @var{TARGET} from @var{SRC}.

@node showmasks, stat, show_explore, Command Reference
@section showmasks
@vskip 5mm
@b{Usage:}
@example

  @t{gxpc} @t{showmasks} [@t{--level} 0/1] [@var{NAME}]
  @t{e}.g.,
  @t{gxpc} @t{showmasks}
  @t{gxpc} @t{showmasks} @t{--level} 1

@end example
@vskip 5mm

@vskip 5mm
@b{Description:}


  Show summary (@t{--level} 0) or detail (@t{--level} 1) of all execution
masks (if `name' is omitted) or a specified excution mask named
`name.' The name of the current default execution mask is `0'.

@vskip 5mm
@b{Examples:}
@example

  @t{gxpc} @t{showmasks}
  @t{gxpc} @t{showmasks} @t{--level} 0     # show summary of all exec masks
  @t{gxpc} @t{showmasks} @t{--level} 0 0   # show summary of the current mask
  @t{gxpc} @t{showmasks} @t{--level} 1 0   # show detail of current exec mask
  @t{gxpc} @t{showmasks} @t{--level} 1 all # show detail of exec mask named `all'

@end example
@vskip 5mm

Without any argument as in the first line, a summary like the
following is shown.

@vskip 5mm
@example

   0 : 1
   1 : 66
   six : 6
   ten : 8
   half : 27

@end example
@vskip 5mm

This says there are three execution masks, called `0', `1',
`six', `ten', and `half.' The right column indicates the number
of nodes that will execute a command when the mask is selected by
a @t{--withmask} (@t{-m}) option. `0' is the current, default execution
mask, used when @t{--withmask} (@t{-m}) option is not given. These
masks are created by @t{savemask} or @t{pushmask} command.

When option @t{--level} 1 is given, details about the execution mask
will be shown, like the following.

@vskip 5mm
@example

  six : 6
   idx: name (gupid)
     -: hongo-lucy-tau-2006-12-31-22-40-24-31387
     -:  hongo002-tau-2006-12-31-13-39-17-338
     0:   hongo010-tau-2006-12-31-22-25-58-20559
     -:  hongo006-tau-2006-12-31-22-27-19-11951
     1:   hongo020-tau-2006-12-31-22-34-06-11026
     -:  hongo001-tau-2006-12-31-22-34-38-15517
     2:   hongo030-tau-2006-12-31-22-44-07-4931
     -:  hongo007-tau-2006-12-31-22-30-54-21738
     3:   hongo040-tau-2006-12-31-22-47-55-32666
     -:  hongo004-tau-2008-12-31-22-42-33-31254
     4:   hongo060-tau-2006-12-31-22-42-59-3899
     -:  hongo005-tau-2006-12-31-22-38-30-1150
     5:   hongo050-tau-2006-12-31-22-40-55-18088

@end example
@vskip 5mm

This is a subtree of the whole tree of gxp daemons.  Nodes that will
actually execute commands will have an index number (0, 1, ..., 5) on
the first column. Nodes marked with `-' will not execute the
command, but are part of the minimum subtree containing those six
nodes.

@vskip 5mm
@b{See Also:}
@example

  @t{smask} @t{rmask} @t{savemask} @t{restoremask} @t{pushmask} @t{popmask}

@end example
@vskip 5mm

@node stat, trim, showmasks, Command Reference
@section stat
@vskip 5mm
@b{Usage:}
@example

  @t{gxpc} @t{stat} [@var{LEVEL}]

@end example
@vskip 5mm

@vskip 5mm
@b{Description:}


  Show all live gxp daemons in tree format. @var{LEVEL} is 0, 1, or 2 and
determines the detail level of the information shown.

@node trim, use, stat, Command Reference
@section trim
@vskip 5mm
@b{Usage:}
@example

  @t{gxpc} @t{trim}

@end example
@vskip 5mm

@vskip 5mm
@b{Description:}


  Trim (release) some subtrees of gxp daemons. This is typically
used after a @t{ping} cmd followed by @t{smask}, to prune non-responding
(dead) daemons.  Specifically, @t{trim} command will be executed on
the selected nodes, and each such node will throw away a children
@var{C} if no nodes under the subtree rooted at @var{C} are selected for
execution of this @t{trim} command. This effectively prunes (trims)
the subtree from the tree of gxp daemons. For example,

@vskip 5mm
@example

  @t{gxpc} @t{ping}
  # If there are some non-responding daemons, this command will hang.
  # type <Ctrl-C> to @t{quit}.
  @t{gxpc} @t{smask}
  @t{gxpc} @t{trim}

@end example
@vskip 5mm

@node use, version, trim, Command Reference
@section use, edges
@vskip 5mm
@b{Usage:}
@example

  @t{gxpc} @t{use}          [@t{--as} @var{USER}] @var{RSH_NAME} @var{SRC} [@var{TARGET}]
  @t{gxpc} @t{use} @t{--delete} [@t{--as} @var{USER}] @var{RSH_NAME} @var{SRC} [@var{TARGET}]
  @t{gxpc} @t{use}
  @t{gxpc} @t{use} @t{--delete} [idx]

@end example
@vskip 5mm

  @t{e}.g.,
  @t{gxpc} @t{use} ssh your_hostname compute_node_prefix

@vskip 5mm
@b{Description:}



@vskip 5mm
@example

  Configure rsh-like commands used to login targets matching a
particular pattern from hosts matching a particular pattern. The
typical usage is `@t{gxpc} @t{use} @var{RSH_NAME} @var{SRC} @var{TARGET}', which says gxp can
@t{use} an rsh-like command @var{RSH_NAME} for @var{SRC} to login @var{TARGET}. @t{gxpc}
remembers these facts to decide which hosts should issue which
commands to login which hosts, when @t{explore} command is issued. See the
tutorial section of the manual.

@end example
@vskip 5mm

@vskip 5mm
@b{Examples:}
@example

  @t{gxpc} @t{use}           ssh abc000.def.com pqr.xyz.ac.jp
  @t{gxpc} @t{use}           ssh abc000 pqr
  @t{gxpc} @t{use}           ssh abc
  @t{gxpc} @t{use}           @t{rsh} abc
  @t{gxpc} @t{use} @t{--as} taue ssh abc000 pqr
  @t{gxpc} @t{use} qrsh      abc
  @t{gxpc} @t{use} qrsh_host abc
  @t{gxpc} @t{use} sge       abc
  @t{gxpc} @t{use} torque    abc

@end example
@vskip 5mm

The first line says that, if @t{gxpc} is told to login pqr.xyz.ac.jp by
@t{explore} command, hosts named abc000.def.com can @t{use} `ssh' method to do
so.  How it translates into the actual ssh command line can be shown
by `@t{show_explore}' command (try `@t{gxpc} @t{help} @t{show_explore}') and can be
configured by `@t{rsh}' command (try `@t{gxpc} @t{help} @t{rsh}').

@var{SRC} and @var{TARGET} are actually regular expressions, so the line like the
first one can often be written like the second one.  The first line
is equivalent to the second line as long as there is only one host
begining with abc000 and there is only one target beginning with pqr.
In general, the specification:

@vskip 5mm
@example

  @t{gxpc} @t{use} @var{RSH_NAME} @var{SRC} @var{TARGET}

@end example
@vskip 5mm

is read: if @t{gxpc} is told to login a target matching regular
expession @var{TARGET}, a host matching regular expression @var{SRC} can @t{use}
@var{RSH_NAME} to do so.

Note that the effect of @t{use} command is @var{NOT} to specify which target
@t{gxpc} should login, but to specify @var{HOW} it can do so, if it is told
to. It is the role of @t{explore} command to specify which target hosts it
should login

If the @var{TARGET} argument is omitted as in the third line, it is
treated as if @var{TARGET} expression is @var{SRC}. That is, the third line
is equivalent to:

@vskip 5mm
@example

  @t{gxpc} @t{use} ssh abc abc

@end example
@vskip 5mm

This is often useful to express that ssh login is possible
between hosts within a single cluster, which typically have a
common prefix in their host names. If the traditional @t{rsh} command
is allowed within a single cluster, the fourth line may be useful
too.

If @t{--as} user option is given, login is issued using an explicit user
name. The fifth line says when gxp attempts to login pqr from abc000,
the explicit user name `taue' should be given. You do not need this as
long as the underlying rsh-like command will complement it by a
configuration file. @t{e}.g., ssh will read ~/.ssh/config to complement
user name used to login a particular host.

qrsh_host uses command qrsh, with an explicit hostname argument
to login a particular host (i.@t{e}., qrsh -l hostname=...).  This is
useful in environments where direct ssh is discouraged or
disallowed and qrsh is preferred.

qrsh also uses qrsh, but without an explicit hostname. The host
is selected by the scheduler. Therefore it does not @t{make} sense to
try to speficify a particular hostname as @var{TARGET}.  Thus, the
effect of the line

@vskip 5mm
@example

  @t{gxpc} @t{use} qrsh abc

@end example
@vskip 5mm

is if targets beginning with abc is given (upon @t{explore} command),
a host beginning with abc will issue qrsh, and get whichever host
is allocated by the scheduler.

@vskip 5mm
@b{See Also:}
@example

  @t{explore} @t{rsh}

@end example
@vskip 5mm

@node version,  , use, Command Reference
@section version
@vskip 5mm
@b{Usage:}
@example

  @t{gxpc} @t{version}

@end example
@vskip 5mm


@node Tools Reference, Troubleshooting, Command Reference, Top
@chapter Tools Reference
todo

@node Troubleshooting, Environment Variables Reference, Tools Reference, Top
@chapter Troubleshooting
todo

@node Environment Variables Reference, Key Stroke Reference, Troubleshooting, Top
@chapter Environment Variables Reference
todo

@node Key Stroke Reference,  , Environment Variables Reference, Top
@chapter Key Stroke Reference
todo

@menu
* Indices::                     
* Function Index::              
* Variable Index::              
* Data Type Index::             
* Program Index::               
* Concept Index::               
@end menu

@node Indices, Function Index, Key Stroke Reference, Key Stroke Reference
@unnumberedsec Indices

@menu
* Function Index::              
* Variable Index::              
* Data Type Index::             
* Program Index::               
* Concept Index::               
@end menu

@node Function Index, Variable Index, Indices, Key Stroke Reference
@unnumberedsec Function Index

@printindex fn

@node Variable Index, Data Type Index, Function Index, Key Stroke Reference
@unnumberedsec Variable Index

@printindex vr

@node Data Type Index, Program Index, Variable Index, Key Stroke Reference
@unnumberedsec Data Type Index

@printindex tp

@node Program Index, Concept Index, Data Type Index, Key Stroke Reference
@unnumberedsec Program Index

@printindex pg

@node Concept Index,  , Program Index, Key Stroke Reference
@unnumberedsec Concept Index

@printindex cp


@contents

@bye
                                   

